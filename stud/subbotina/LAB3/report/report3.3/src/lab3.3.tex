\section* {3.3  Нахождение приближающих многочленов 1-ой и 2-ой степени}

\subsection{Постановка задачи}
Для таблично заданной функции путем решения нормальной системы МНК найти приближающие многочлены a) 1-ой и б) 2-ой степени. Для каждого из приближающих многочленов вычислить сумму квадратов ошибок. Построить графики приближаемой функции и приближающих многочленов.
 

{\bfseries Вариант:} 22

\begin{figure}[h!]
\centering
\includegraphics[width=15cm, height=4cm]{usl2}
\caption{Условия}
\end{figure}
% \pagebreak

\subsection{Результаты работы}
\begin{figure}[h!]
\centering
\includegraphics[width=.9\textwidth]{lab3.3}
\caption{Вывод программы в консоли}
\end{figure}

% \vfill

% \begin{figure}[h!]
% \centering
% \includegraphics[width=.9\textwidth]{lab5_taylor}
% \caption{Решение с аппроксимацией граничных условий со вторым порядком}
% \end{figure}
\pagebreak

\subsection{Исходный код}
% \lstinputlisting[language=C++]{matrix.cpp}
% \begin{lstlisting}
\lstinputlisting{include/lab3_3.cpp}
\lstinputlisting{include/matrix.cpp}
\lstinputlisting{include/MNK.h}
\lstinputlisting{include/polynom.h}
% \end{lstlisting}
% \lstinputlisting{matrix.cpp}
% {../../include/matrix.cpp}
% \pagebreak
% \lstinputlisting[title=\texttt{parabolic\_pde.hpp}]{../../include/partial_differential/parabolic_pde.hpp}
% \pagebreak
% 